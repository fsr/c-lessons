\input{slides_template}	% nothing to do here
\usepackage[english]{babel}

\usepackage[utf8]{inputenc}

\newcommand{\course}{
	C introduction
}

\author{
	Richard Mörbitz,
	Manuel Thieme
}

\lstset{
	language = C,
	showspaces = false,
	showtabs = false,
	showstringspaces = false,
	tabsize = 4,
	escapechar = @
} % TODO modify this if you have not already done so

% meta-information
\newcommand{\topic}{
	Operators
}

% nothing to do here
\title{\topic}
\supertitle{\course}
\date{\today}

% the actual document
\begin{document}

\maketitle

\begin{frame}{Contents}
	\tableofcontents
\end{frame}

\section{Overview}
\subsection{}
\begin{frame}[fragile]{Let's calc}
	Each time you type in a value or a variable, it gets evaluated as an expression. You can use operators to make new expressions out of existing ones.
	\begin{lstlisting}[numbers=none]
3;				/* value 3 */
int a = 5; 		/* value 5 */
int b = a + 3;	/* value 8 */
\end{lstlisting}
Got the idea?\\
\ \\Short: Operators are for calculations, comparisons and more.
\end{frame}
\section{Arithmetic operators}
\subsection{}
\begin{frame}[fragile]{Back to School}
	\begin{itemize}
		\item +, -, *, / as all of you should know
		\item \% is the modulo (remainder) operator
		\item *, /, \% get evaluated before +, -
		\item operations in (  ) are of higher precedence
	\end{itemize}
	\ \\\ \\Something to try out and think about:
	\begin{lstlisting}[numbers=none]
char a = 120;	/* 120 */
2 * a;			/* not 240 */
\end{lstlisting}
\end{frame}
\begin{frame}[fragile]{You see, it's not that easy}
	\begin{itemize}
		\item You shall not devide integers by zero
		\item Integer division differs from floating point division
		\item You can use operators between different data types
		\begin{itemize}
			\item[--] mixing different sizes
			\item[--] mixing integer and floating point variables
		\end{itemize}
	\end{itemize}
	\ \\\ \\Try it yourself.
	\begin{lstlisting}[numbers=none]
int i1 = 42, i2 = 23;
short s = 13;
float f = 3.14;

i1 / i2;	/* results in 1, not a real division */
i1 + s;		/* int and short, result is int */
i1 / f;		/* result is float, actual division */
\end{lstlisting}
\end{frame}
\section{Relational operators}
\subsection{}
\begin{frame}{The truth about expressions}
	Expressions can also be evaluated to truth values.\\
	If a value or a variable equals 0, its corresponding truth value is \textit{false}. Otherwise it's \textit{true}.\\
	The representations of \textit{true} and \textit{false} are 0 and 1.\\
	An expression containing relational operators gets evaluated to such a truth value.\\
	\ \\Relational operators:
	\begin{itemize}
		\item $<$, $>$, $<$=, $>$=
		\item == for "equal to"
		\item != for "not equal to"
	\end{itemize}
\end{frame}
\begin{frame}[fragile]{Do not get confused}
	Imagine the following
	\begin{lstlisting}[numbers=none]
(5 < 7) =@\,@= 1;	/* evaluated to 1 */
\end{lstlisting}
\ \\\ \\Why?
\begin{itemize}
	\item<2-> (5 $<$ 7) is \textbf{true} $\rightarrow$ \textbf{1}
	\item<3-> 1 == 1 is \textbf{true} $\rightarrow$ \textbf{1}
\end{itemize}
\end{frame}
\section{Logical operators}
\subsection{}
\begin{frame}{Boolean arithmetic}
	Truth values can be connected by boolean operators resulting in a new truth value.
	\begin{itemize}
		\item \&\& for AND (results in \textbf{1} if both operands are true, else \textbf{0})
		\item $||$ for OR (results in \textbf{1} if at least one operator is true, else \textbf{0})
		\item ! for NOT (results in \textbf{1} if the operand is false, else \textbf{0})
	\end{itemize}
	\ \\\ \\Precedence order:\\
	\centering
	! \ $>$ \ \&\& \ $>$ \ $||$ 
\end{frame}
\begin{frame}[fragile]{Seems logic}
	How do you get NAND, NOR and XOR?
	\begin{uncoverenv}<2->
		\begin{lstlisting}[numbers=none]
int a, b;
...
!(a && b);	/* NAND */
!(a || b);	/* NOR */
a != b;		/* XOR */
\end{lstlisting}
	\end{uncoverenv}
\end{frame}
\section{More on assignments}
\subsection{}
\begin{frame}[fragile]{A sign meant...}
	We already know the assignment operator.\\
	Assignments are expressions, too, and expressions can be assigned to variables.\\\ \\
	The general syntax is:	\textit{$<$identifier$>$ = $<$expression$>$}\\\ \\
	
	\begin{uncoverenv}<2->
		Consider:
		\begin{lstlisting}[numbers=none]
c = 0;				/* false */
c = 2 * 5;			/* (2 * 5) evaluates to 10, c is 10 */
c = (0 < 1);		/* (0 < 1) evaluates to true, c is 1 */

a = (b =@\,@= (c = d));	/* Wat? */
\end{lstlisting}
	\end{uncoverenv}
\end{frame}
\begin{frame}[fragile]{Syntatic sugar}
	It is common that the expression on the right side of an assignment contains the variable it is assigned to.\\
	To avoid redundancy, C offers the following short forms:\\\ \\
	\begin{lstlisting}[numbers=none]
a = a + 4;	/* old and busted */
a += 4;		/* new hotness, a = a + 4; */
a -= 4;		/* a = a - 4; */
a *= b;		/* a = a * b; */
c /= 42;	/* c = c / 42; */
d %= 2;		/* d = d % 2; */
\end{lstlisting}
\end{frame}
\begin{frame}[fragile]{C++}
	To increment (+1) and decrement (-1) a variable, there is an even shorter shortform:
	\begin{lstlisting}[numbers=none]
int C = 42;
C++;	/* C += 1; */
++C;	/* C += 1; */
C--;	/* C -= 1; */
\end{lstlisting}
\ \\\ \\C++ expressions are evaluated before the increment while ++C increments first (the same applies on C$--$):
	\begin{lstlisting}[numbers=none]
int C = 42;
int a = C++;	/* a is 42 */
int b = ++C;	/* b is 44 */
\end{lstlisting}
\end{frame}
\begin{frame}{The stage is yours}
	\begin{itemize}
		\item Write a Program that asks the user for two decimal numbers and prints their sum, difference, product, integer quotient and remainder.
		\item Write a Program that asks the user for 5 characters and prints them in reversed order.
		\item Write a Program that calculates the area of a circle.\\(Let's assume $\pi$ == 3.14)
	\end{itemize}
\end{frame}
% nothing to do from here on
\end{document}
