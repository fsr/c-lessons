\input{slides_template}	% nothing to do here
\input{c_advanced_info} % TODO modify this if you have not already done so

% meta-information
\newcommand{\topic}{%
    Network Programming
}

% nothing to do here
\title{\topic}
\supertitle{\course}
\date{}

% the actual document
\begin{document}

\maketitle

\begin{frame}{Contents}
	\tableofcontents
\end{frame}

\section{Network Protocols}
\subsection{}

\begin{frame}[Osi Reference Model]
    % OSI model, Pfeil auf Layer 4
\end{frame}

\begin{frame}[Transport Protocols]
    TCP:
    \begin{itemize}
            % bissl Foo
    \end{itemize}
    UDP:
    \begin{itemize}
            % bissl Bar
    \end{itemize}
    % vielleicht auch keine zwei itemize, sondern was schöneres
\end{frame}

\section{Socket Programming}
\subsection{}

\begin{frame}[Sockets]
    % Erklärungsfoo
    % http://openbook.rheinwerk-verlag.de/c_von_a_bis_z/025_c_netzwerkprogrammierung_001.htm#mj026dcf95963c30f3a232837965a8e774
    % Unix: files (int - file descriptor), Windows: objects (SOCKET - handle)
    % und die beiden Header
\end{frame}

% Dann noch jeweils ein Frame zu folgenden funktionen mit kurzer Erklärung
% sowie Windows- und Unix-Prototyp
% Dabei die Argumente und einsetzbaren Konstanten erklären
% - socket()
% -- hier evtl. noch ein Frame mit Erklärung der Netzwerkadressrechnung
% - close[socket]()
% - connect()
% - bind()
% - listen()
% - accept()
% - send() / recv() / sendto() / recvfrom()

% nothing to do from here on
\end{document}
