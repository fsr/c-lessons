\input{slides_template}	% nothing to do here
\input{c_advanced_info}

% meta-information
\newcommand{\topic}{
	Debugging
}

% nothing to do here
\title{\topic}
\supertitle{\course}
\date{}

% the actual document
\begin{document}

\maketitle

\begin{frame}{Contents}
	\tableofcontents
\end{frame}

\section{Bugs}
\subsection{}
\begin{frame}{It's not a bug...}
	There are different kinds of errors.
	\begin{itemize}
		\item Compilitime errors
		\item Runtime errors (\textit{Bugs})
	\end{itemize}\ \\\ \\
	\textit{Compiletime errors} are easily handable since the compiler shows you where to fix them.\\\ \\
	\textit{Bugs} on the other hand are harder to find because you have no idea where to look for them.
\end{frame}
\begin{frame}{... it's a feature.}
	Bugs can appear due to different reasons
	\begin{itemize}
		\item Variable overflow
		\item Division by Zero
		\item Infinite loops / recursions
		\item Range excess
		\item Segmentation fault
		\item Dereferencing \textit{NULL pointers}
		\item ...
	\end{itemize}
\end{frame}
\begin{frame}{The dungeon}
	We prepared a little ASCII dungeon.\\
	You can find it in the repository (\url{https://github.com/fsr/c-slides}) in folder \textit{materials/1\_before/}\\
	\begin{itemize}
		\item Look at the code and try to understand what should happen.
		\item If you find mistakes, please leave them. We'll fix them later.
		\\\ 
		\item Compile it (with \textit{-std=c99})
		\\\ 
		\item And now run it.
	\end{itemize}
	
\end{frame}
\section{Warnings}
\subsection{}
\begin{frame}{\textit{gcc} flags}
	You can instruct \textit{gcc} to display different warning levels and even abort compiling when warnings appear.
	These instructions are called flags.\\\ \\\ \\
	\begin{tabular}{|l|l|}\hline
		\textit{-Wall} & enables a bunch of compiler warnings\\\hline
		\textit{-Wextra} & enables further compiler warnings\\\hline
		\textit{-Werror} & warnings are interpreted as errors (compiling doesn't succeed)\\\hline
	\end{tabular}
	\ \\\ \\\ \\
	\begin{itemize}
		\item You can pass an arbitrary amount of flags
		\item Seperate them with whitespaces
	\end{itemize}
\end{frame}
\begin{frame}{Debugging light}
	\begin{itemize}
		\item Compile the dungeon with \textit{-Wall} and \textit{-Wextra}
		\begin{itemize}		
			\item There are 3 Warnings. The 3rd one matters.
			\item There is a variable that's set but not used.
		\end{itemize}
		\item You may find a bug there. Fix it.
	\end{itemize}
\end{frame}
\section{GDB}
\subsection{}
\begin{frame}[fragile]{The \textbf{G}NU \textbf{D}e\textbf{B}ugger}
	There are tools helping with bugs, called debuggers. GDB is one of them.
	To Use it
	\begin{itemize}
		\item You have to install the package \textit{gdb}\\
		On Windows, hope your cygwin installation came with it
		\item You have to compile your program with the \textit{-g} flag
		\begin{lstlisting}[numbers=none]
$ gcc -g main.c
\end{lstlisting}
		\item After that you can start your program with gdb:
		\begin{lstlisting}[numbers=none]
$ gdb a.out
\end{lstlisting}
	\end{itemize}
\end{frame}
\begin{frame}{Commands}
	\begin{itemize}
		\item If you started gdb without a file you can load it with \textbf{file} \textit{file\_name}.
		\item Use \textbf{r[un]} to execute the program with gdb.\\
		You should begin with it. It will give you further information about the crash.
		\item You can set an arbitrary amount of breakpoints with \textbf{b[reak]} \textit{line\_number} or \textbf{b[reak]} \textit{function\_name}.\\
		Begin with a breakpoint at the point the program crashes.
		\item Print values with \textbf{p[rint]} \textit{identifier}.
		\item Use \textbf{w[atch]} \textit{identifier} to break and print a variable when it's changed.
	\end{itemize}
\end{frame}
\begin{frame}{Once you're at a breakpoint}
	\begin{itemize}
		\item Use \textbf{n[ext]} to execute the next program line only.
		\item \textbf{s[tep]} executes the next instruction.
		\item You can jump to the next breakpoint with \textbf{c[ontinue]}.
		\item To see How you have come to this point in the program flow, type \textbf{backtrace} or \textbf{bt}.\\
		This shows you all functions you called to come there.
		\\\ \\
		\item By only hitting the \textit{return key}, you repeat the last entered  command.
	\end{itemize}
	\ \\\ \\
	GDB is much more mighty than this few commands, but that should be sufficient to solv your final quest.
\end{frame}
\begin{frame}{Now it's up to you}
	\begin{itemize}
		\item Find and fix all Bugs in the dungeon.\\\ \\
		\begin{tabular}{|l|l|}
			\hline
			\textbf{file} & load program\\\hline
			\textbf{r[un]} & execute program\\\hline
			\textbf{b[reak]} & set breakpoint\\\hline
			\textbf{p[rint]} & print variable\\\hline
			\textbf{w[atch]} & break and print variable when it changes\\\hline
			\textbf{n[ext]} & execute next line and break\\\hline
			\textbf{s[tep]} & execute next instruction and break\\\hline
			\textbf{c[ontinue]} & execute until next breakpoint\\\hline
			\textbf{backtrace} / \textbf{bt} & How did I end up here?\\\hline
		\end{tabular}
	\end{itemize}
\end{frame}
% nothing to do from here on
\end{document}
