\input{slides_template}	% nothing to do here
\input{c_advanced_info} % TODO modify this if you have not already done so

% meta-information
\newcommand{\topic}{
	Project organisation and external libraries
}

% nothing to do here
\title{\topic}
\supertitle{\course}
\date{}

% the actual document
\begin{document}

\maketitle

\begin{frame}{Contents}
	\tableofcontents
\end{frame}

\section{Header files}
\subsection{}
\begin{frame}{It is a mess}
	The current version of our Dungeon is \textit{2\_finished/main.c}. In this file, there are:
	\begin{itemize}
		\item 205 lines of code
		\item 4 includes
		\item 2 custom enums
		\item 3 custom data type definitions
		\item 4 global variables
		\item 14 functions and its prototypes
		\item And the main function
	\end{itemize}\ \\\ \\
	We now will organize the Project in multiple files.
\end{frame}

\begin{frame}[fragile]{Header files}
	The first thing we do is to seperate the I/O part into a file called \textit{io.c}:\\
	\begin{itemize}
		\item \textit{handle\_input, draw\_board, print\_entity}
	\end{itemize}
	Since we are calling the I/O functions in our \textit{main.c} file, the prototypes have to stay accessible from there.\\
	This is why we are creating our own \textbf{header file} \textit{io.h}, containing the prototypes, and we include it in our \textit{main.c} and \textit{io.c}:\\\ \\
	\begin{lstlisting}
#include "io.h"
\end{lstlisting}\ \\\ \\
	\begin{itemize}
		\item Note that we use \textbf{"\ldots"} for own includes instead of $\boldsymbol{<\ldots>}$.
		\item The quotation marks contain the relative path to the header file.
	\end{itemize}
\end{frame}

\begin{frame}{Data structures}
	The next step is to seperate all functions concerning the custom data structures to \textit{datastructures.c}:
	\begin{itemize}
		\item init\_entity
		\item init\_monster\_list
		\item free\_monster\_list
		\item add\_monster
		\item monster\_list\_map
		\item monster\_list\_map\_and\_check
		\item position\_covered
		\item has\_position
	\end{itemize}\ \\\ \\
	We put the type definition of our structs with the prototypes into the header file \textit{datastructures.h}.

\end{frame}

\begin{frame}[fragile]{Global variables}
	Since the source files all are accessing the global variables, we move them into another header file \textit{globals.h}.\\\ \\
	Finally we outsource the function \textit{out\_of\_bounds} and the global variable \textit{next\_coords} into \textit{board.c}.\\
	\begin{itemize}
		\item Try to find the remaining includes by yourself.
	\end{itemize}\ \\\ \\
	To avoid multiple includes of one header file we can, and should, do as follows:
	\begin{lstlisting}
#ifndef BOARD_H
#define BOARD_H

/* header file content here */

#endif /* BOARD_H */
\end{lstlisting}
\end{frame}

\section{Make}
\subsection{}
\begin{frame}[fragile]{Compiling}
	To compile a project from multiple files you have to pass all the source files to the gcc command:\\\ \\
	\begin{lstlisting}[numbers=none]
$ gcc -Wall -o main datastructures.c io.c board.c main.c
\end{lstlisting}\ \\\ \\
	\begin{itemize}
		\item Note that you do not have to add the header files.
		\item They do not need to be compiled separately.
	\end{itemize}
\end{frame}

\begin{frame}[fragile]{Again and again}
	Consider a project that needs to be linked against the standard math library and a graphical one. Consider it having more than just one error flag and more than 10 source files:\\\ \\
	\begin{lstlisting}[numbers=none]
$ gcc -Wall -Wextra -Werror -Wmissing-declarations -g -O3 -o main firstfile.c sencondfile.c thirdfile.c fourthfile.c fifthfile.c sixthfile.c seventhfile.c eighthfile.c ninthfile.c tenthfile.c main.c -lm -lSDL -lSDL2
\end{lstlisting}\ \\\ \\
	It is a mess to type this command everytime you want to compile your program.\\
	If only...
\end{frame}

\begin{frame}[fragile]{Makefiles}
	There is a package called \textbf{make} that allows you to write the compile order into a file and start compiling with a more simple command:\\\ \\
	\begin{lstlisting}[numbers=none]
$ make
\end{lstlisting}\ \\\ \\
	To use it, you have to install the package and create a file named \textit{Makefile} in your project directory.\\
	\begin{itemize}
		\item Note that the \textbf{make} package is much more mighty than that.
	\end{itemize}
\end{frame}

\begin{frame}[fragile]{Make rulez!}
	You can define rules in your makefile:
	\begin{lstlisting}[language=make,basicstyle=\scriptsize,numbers=none]
all:
	gcc -o main main.c
debug:
	gcc -Wall -Werror -Wextra -g -o main main.c
\end{lstlisting}
	Now you can Build your project with:
	\begin{lstlisting}[basicstyle=\scriptsize,numbers=none]
$ make all
\end{lstlisting}
	or with
	\begin{lstlisting}[basicstyle=\scriptsize,numbers=none]
$ make debug
\end{lstlisting}
	if you want to have errors and debug information.
	\begin{itemize}
		\item Note that if you are calling make without a rule, the first one is executed.
	\end{itemize}
\end{frame}

\begin{frame}{About compiling}
	With many source files, we do not want to recompile the whole project if we changed a tiny piece of code.\\
	Make offers the opportunity to compile only if dependecies have changed. But first a little theory about compiling:\\\ \\
	\scriptsize
	\centering
	\begin{tabular}{|l|l|l|l|}
	\hline
	\textbf{process}	&	\textbf{input}	&	\textbf{output}										&	\textbf{notes}				\\\hline
	preprocessor		&	\textit{*.c}	&	source code with substitutions \textit{*.i}			&								\\\hline
	compiling			&	\textit{*.i}	&	assembler mnemonics \textit{*.s}					&	debug information is added	\\\hline
	assembling			&	\textit{*.s}	&	byte code \textit{*.o}								&	not executable yet			\\\hline
	linking				&	\textit{*.o}	&	executable byte code \textit{*} / \textit{*.exe}	&								\\\hline				
	\end{tabular}\ \\\ \\
	\flushleft
	\normalsize
	We would like to split the compilation process and keep the unlinked *.o files to link newly build *.o files with existing ones.
\end{frame}

\begin{frame}[fragile]{Dependecies}
	gcc compiles without linking when the \textit{-c} flag is set.\\
	You can add a dependency to your makefile by writing it behind the rule identifier. Dependencies on other rules are allowed:\\\ \\
	\begin{lstlisting}[language=make,basicstyle=\scriptsize,numbers=none]
all: firstfile.o secondfile.o 
	gcc -o main firstfile.o secondfile.o

firstfile.o: firstfile.c secondfile.h
	gcc -c firstfile.c
	
secondfile.o: secondfile.c secondfile.h
	gcc -c secondfile.c
\end{lstlisting}
	\begin{itemize}
		\item Note that both source files have the same header file as a dependency here.
	\end{itemize}
\end{frame}

\begin{frame}[fragile]{A few tutorials later}
	\begin{lstlisting}[language=make,basicstyle=\tiny,escapeinside=§]
TARGET =main
SOURCES =datastructures.c io.c main.c
OBJECTS =$(SOURCES:.c=.o)
DEPS =$(SOURCES:.c=.d)
CFLAGS =-Wall -Wextra -Werror -Wmissing-declarations -O3 -std=c99

all: $(TARGET)

debug: CFLAGS += -g
debug: $(TARGET)

$(TARGET): $(OBJECTS)
	$(CC) -o $§@§ $^

%.o: %.c Makefile
	$(CC) -MMD -c $(CFLAGS) -o $§@§ $<

.PHONY : clean
clean:
	rm -f $(TARGET) $(OBJECTS) $(DEPS)

-include $(DEPS)
\end{lstlisting}
\end{frame}

\section{ncurses}
\subsection{}

\begin{frame}{An external library}
	The next step to build an awesome ASCII Dungeon is to get rid of the "enter mashing" after entering the direction.\\\ \\
	There is no way to do this cross platform with the standard library. That is why we will use an external one: \textbf{ncurses}.\\\ \\
	We prepared this for you in \textit{3\_finished/io.c}. If you want, you can try to understand the content of that file.
	\begin{itemize}
		\item Have a closer look at \url{https://de.wikibooks.org/wiki/Ncurses}
	\end{itemize}

\end{frame}

% nothing to do from here on
\end{document}
