\input{slides_template}	% nothing to do here
\input{c_advanced_info} % TODO modify this if you have not already done so

% meta-information
\newcommand{\topic}{
	File management
}

% nothing to do here
\title{\topic}
\supertitle{\course}
\date{}

% the actual document
\begin{document}

\maketitle

\begin{frame}{Contents}
	\tableofcontents
\end{frame}

\section{Filedescriptors}
\subsection{}

\begin{frame}[fragile]{Describing a file}
	The first way to handle files in C is to code very low level by using \textbf{filedescriptors}.\\
	Those are Integer numbers that represent a file.\\
	To open a file and get a descriptor there is the system call function \textit{open()}:\\\ \\

	\begin{lstlisting}
int myfile = open("savestate", O_RDWR);
\end{lstlisting}

	\begin{itemize}
		\item See \textit{man 2 open} for further information and library includes concerning this function.
	\end{itemize}

\end{frame}

\begin{frame}[fragile]{Write and read}
	For actually putting data into a file, C offers the function \textit{write()}, that takes a file descriptor, the input and the amount of Bytes that should be written:\\\ \\
	
	\begin{lstlisting}
int myfile = open("savestate", O_WRONLY | O_CREAT, 0644);
char* mytext = "foo";
write(myfile, mytext, 4);
\end{lstlisting}\ \\\ \\

	To read data from a file, you have to prepare a variable to put it in:\\\ \\
	
	\begin{lstlisting}
int myfile = open("savestate", O_RDONLY);
char* mytext[256] = {0};
read(myfile, &mytext, 255);
\end{lstlisting}
	
\end{frame}

\begin{frame}[fragile]{Close}
	After finishing your file operations, you have to tell your OS that you have finished by closing the filedescriptor:\\\ \\
	
	\begin{lstlisting}
int myfile = open("savestate", O_WRONLY | O_CREAT, 0644);
char* mytext = "foo";
write(myfile, mytext, 4);
close(myfile);
\end{lstlisting}
\end{frame}

\section{Filepointers}
\subsection{}

\begin{frame}[fragile]{Returning to standard}
	The C standard library contains a special datastructure for handling files: a so called \textbf{filepointer} \textit{FILE*}:\\\ \\
	
	\begin{lstlisting}
FILE* myfile = fopen("savestate", "r+");
char* content[256] = {0};
fgets(&content, 255, myfile);
fputs("foo", myfile);
fclose(myfile);
\end{lstlisting}

	\begin{itemize}
		\item See the man page for further information.
	\end{itemize}
\end{frame}

\section{Dungeon}
\subsection{}
\begin{frame}{Savestates}
	Now that you are common with file management, implement a savestate functionality into the dungeon.\\\ \\
	\begin{itemize}
		\item When the game quits (by pressing \textit{x}) the game data should be written into a savestate.
		\item If there is a savestate, the game should load it on start.
		\item There also should be an opportunity to exit the game without saving.
	\end{itemize}
\end{frame}

% nothing to do from here on
\end{document}
