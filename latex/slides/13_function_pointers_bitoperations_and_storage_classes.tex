\input{slides_template}	% nothing to do here
\input{c_advanced_info} % TODO modify this if you have not already done so

% meta-information
\newcommand{\topic}{
	Bit operations
}

% nothing to do here
\title{\topic}
\supertitle{\course}
\date{}

% the actual document
\begin{document}

\maketitle

\begin{frame}{Contents}
	\tableofcontents
\end{frame}

\section{Bit operations}
\subsection{}

\begin{frame}{A little bit of logic}
	As you know, all data is stored as \textit{binary numbers} - sequences of 0 and 1.\\
	In C, you can operate on this bit layer by using the following \textit{logical bit} and \textit{shift} operators:\bigskip
	
	\begin{tabular}{|c|c|l|}
																						  	  \hline
		\textbf{Symbol} 	& \textbf{Operation} 	& \textbf{Example} 							\\\hline
		$|$					& logical or				& $0110\ |\ 0101 == 0111$ 		\\\hline
		$\&$ 				& logical and 				& $0110\ \&\ 0101 == 0100$ 	\\\hline
		\textasciicircum				& logical xor 				& $0110\ $\textasciicircum\ $0101 == 0011$ 	\\\hline
		\textasciitilde			& logical negation 			& \textasciitilde $0110 == 1001$	\\\hline
		$<<$ 			& shift to the left 			& $0110\ <<\ 2 == 011000$ 	\\\hline
		$>>$ 		& shift to the right 				& $0110\ >>\ 2 == 0001$ 			\\\hline
	\end{tabular}
	
\end{frame}

\begin{frame}[fragile]{Computational arithmetics}
	With bit operations, some mathematical tasks can be solved more efficiently:
	\begin{itemize}
		\item Multiplying/dividing by $2^n$ is equivalent to a shift by $n$ bits
		\begin{lstlisting}
5 * 8 == 5 << 3;
60 / 4 == 60 >> 2;
\end{lstlisting}
		\item Instead of $\%\ 2^n$ you can use $\&\ 2^{n-1}$ 
		\begin{lstlisting}
22 % 2 == 22 & 1;
24 % 16 == 24 & 15;
\end{lstlisting}
	\end{itemize}\bigskip
	Be aware of the fact that the readability of your code will suffer from that. Most of these optimisations are done by the compiler anyway.
\end{frame}
\begin{frame}[fragile]{Masking}
	$x\ |\ mask$ sets all bits in $x$ that are $1$ in $mask$.\\
	\begin{lstlisting}
'A' | 32;	/* Let a capital letter be small */
\end{lstlisting}\bigskip
	$x\ \&\ mask$ deletes all bits in $x$ that are $0$ in $mask$.\\
	\begin{lstlisting}
'a' & ~32;	/* Let a small letter be capital */
\end{lstlisting}\bigskip
	$x\ $\textasciicircum\ $mask$ inverts all bits in $x$ that are $1$ in $mask$.\\
	\begin{lstlisting}
'a' ^ 32;	/* "Toggle" a letter */
\end{lstlisting}
\end{frame}

\begin{frame}[fragile]{Bit fields}
	Although it may seem efficient to use each bit of a number to store information in it, it will become nasty to access all the values by:\\
	$x\ \&\ 1$, $x\ \&\ 2$, $x\ \&\ 4$, \dots all the way up to $2^{sizeof\ int - 1}$\\
	\bigskip
	For this particular reason, C offers \textit{bit fields} like the following:
	\begin{lstlisting}
struct traffic_light {
	int red		: 1;
	int yellow	: 1;
	int green	: 1;
	int			: 5;	/* not in use */
};
\end{lstlisting}
	The members of bit fields can be accessed as if they were members of ordinary \textit{struct}s.

\end{frame}

% nothing to do from here on
\end{document}
