\input{slides_template}	% nothing to do here
\input{c_advanced_info} % TODO modify this if you have not already done so

% meta-information
\newcommand{\topic}{
	Bitoperations, storage classes, macros and function pointers
}

% nothing to do here
\title{\topic}
\supertitle{\course}
\date{}

% the actual document
\begin{document}

\maketitle

\begin{frame}{Contents}
	\tableofcontents
\end{frame}

\section{Bitoperations}
\subsection{}

\begin{frame}{A little bit of logic}
	As you know, all data is stored as \textit{binary numbers} - sequences of 0 and 1.\\
	In C, you can operate on this bit layer by using the following \textit{logical bit} and \textit{shift} operators:\bigskip
	
	\begin{tabular}{|c|c|l|}
																						  	  \hline
		\textbf{Symbol} 	& \textbf{Operation} 	& \textbf{Example} 							\\\hline
		$|$					& logical or				& $0110\ |\ 0101 == 0111$ 		\\\hline
		$\&$ 				& logical and 				& $0110\ \&\ 0101 == 0100$ 	\\\hline
		\textasciicircum				& logical xor 				& $0110\ $\textasciicircum\ $0101 == 0011$ 	\\\hline
		\textasciitilde			& logical negation 			& \textasciitilde $0110 == 1001$	\\\hline
		$<<$ 			& shift to the left 			& $0110\ <<\ 2 == 011000$ 	\\\hline
		$>>$ 		& shift to the right 				& $0110\ >>\ 2 == 0001$ 			\\\hline
	\end{tabular}
	
\end{frame}

\begin{frame}[fragile]{Computational arithmetics}
	With bit operations, some mathematical tasks can be solved more efficiently:
	\begin{itemize}
		\item Multiplying/dividing by $2^n$ is equivalent to a shift by $n$ bits
		\begin{lstlisting}
5 * 8 == 5 << 3;
60 / 4 == 60 >> 2;
\end{lstlisting}
		\item Instead of $\%\ 2^n$ you can use $\&\ 2^{n-1}$ 
		\begin{lstlisting}
22 % 2 == 22 & 1;
24 % 16 == 24 & 15;
\end{lstlisting}
	\end{itemize}\bigskip
	Be aware of the fact that the readability of your code will suffer from that. Most of these optimisations are done by the compiler anyway.
\end{frame}
\begin{frame}[fragile]{Masking}
	$x\ |\ mask$ sets all bits in $x$ that are $1$ in $mask$.\\
	\begin{lstlisting}
'A' | 32;	/* Make a capital letter be small */
\end{lstlisting}\bigskip
	$x\ \&\ mask$ deletes all bits in $x$ that are $0$ in $mask$.\\
	\begin{lstlisting}
'a' & ~32;	/* Make a small letter be capital */
\end{lstlisting}\bigskip
	$x\ $\textasciicircum\ $mask$ inverts all bits in $x$ that are $1$ in $mask$.\\
	\begin{lstlisting}
'a' ^ 32;	/* "Toggle" a letter */
\end{lstlisting}
\end{frame}

\begin{frame}[fragile]{Bit fields}
	Although it may seem efficient to use each bit of a number to store information in it, it will become nasty to access all the values by:\\
	$x\ \&\ 1$, $x\ \&\ 2$, $x\ \&\ 4$, \dots all the way up to $2^{INT\_MAX - 1}$\\
	\bigskip
	For this particular reason, C offers \textit{bit fields} like the following:
	\begin{lstlisting}
struct traffic_light {
	int red		: 1;
	int yellow	: 1;
	int green	: 1;
	int			: 5;	/* not in use */
};
\end{lstlisting}
	The members of bit fields can be accessed as if they were members of ordinary \textit{struct}s.

\end{frame}

\section{Storage classes}
\subsection{}

\begin{frame}{Back to scopes}
	By introducing programs consisting of multiple files, we must differantiate between three cases to determine a variable's scope:
	\begin{enumerate}
		\item \textit{local}: only available in the block of its declaration
		\item \textit{global wrt. a file}: available to all functions in the file of its declaration
		\item \textit{global wrt. the program}: available to all functions in the program
	\end{enumerate}\bigskip
	While local variables get destroyed each time the program leaves their scope, all other variables are alive during the whole program execution.\\ \bigskip
	To modify this behaviour, C allows variable declarations with one of the following keywords: \textit{extern}, \textit{static}, \textit{register} and \textit{auto}.
\end{frame}

\begin{frame}[fragile]{\textit{extern}}
	In order to access a variable that has been defined outside the current file, you have to tell the compiler to look for it in another file.\\
	This is done by prefixing the variable declaration with the keyword \textit{extern}.\\
	\bigskip
	File io.c:
	\begin{lstlisting}
char line_buffer[512];
/* Some basic i/o functions */
\end{lstlisting}
	File parse.c:
	\begin{lstlisting}
extern char line_buffer[];
int parse_line(void) {
	/* Read from line_buffer in here */
}
\end{lstlisting}

\end{frame}

\begin{frame}{\textit{static}}
	\textit{Static} variables are placed on the data segment as if they were global.\\
	Their scope, however, depends on where they are declared:\\
	\bigskip
	\textit{Outside} any function
	\begin{itemize}
		\item They are global in the current file and cannot be accessed from anywhere else
		\item Thus you can have "private" variables and variables with the same names in different files 
	\end{itemize}
	\textit{Inside} a block
	\begin{itemize}
		\item They are local, but do not get destroyed when the block is left
		\item Thus you get functions with a "memory"
	\end{itemize}
\end{frame}

\begin{frame}[fragile]{\textit{register} and \textit{auto}}
	Variables that should be accessed fast can be put in \textit{register}s instead of the memory. However,
	\begin{itemize}
		\item They must not be larger than a word in your architecture
		\item You must not take addresses of such variables
		\item There is only a limited amount of registers, so the compiler may put the variables on the stack ignoring the \textit{register} keyword
	\end{itemize}
	\begin{lstlisting}
for (register i; i < 4096; ++i);	/* int is default */
\end{lstlisting}
	The compiler would do this during optimisation anyway.\\
	\bigskip
	All variables declared inside functions without any special keyword are \textit{auto}matically put on the stack $-$ no need to explicitely use \textit{auto}.
\end{frame}

\begin{frame}[fragile]{Storage classes of functions}
	By default, functions are global and available in the whole program.\\
	\bigskip
	You could declare them \textit{extern} in every file you want to call them, but it is better practice to include a header containing the function declaration.\\
	\bigskip
	Just like variables, \textit{static} functions have a "private" character and can be used to forbid calls from other files and avoid name clashes.
	\begin{lstlisting}
static void puts(char *str) {
	printf("\n%s\n", str);
}
\end{lstlisting}
	\bigskip
	It should be obvious that functions cannot be put into registers.	
\end{frame}

\section{Pointers to functions}
\subsection{}

\begin{frame}[fragile]{Higher order functions}
	It is possible to pass a function as an argument of another function.\\
	Doing so, you get a function that can call different other functions.\\
	\bigskip
	Since functions are placed in the text segment of you program, you simply pass the address of the function as a so called function pointer:\\
	\begin{lstlisting}
<return type> (*<function name>)(<parameter list>);
\end{lstlisting}
	\bigskip
	Example:
	\begin{lstlisting}
int (*op)(int a, int b) = add; /* Defined somewhere else */
int (*print)(const char *) = puts;	/* From stdlib */
\end{lstlisting}
	Note that the $\&$ operator can be omitted.
\end{frame}

\begin{frame}[fragile]{Example}
	\begin{lstlisting}
#include <stdio.h>

int add(int a, int b) {
	return a + b;
}

int sub(int a, int b) {
	return a - b;
}

void printFunc(int (*f)(int, int), int a, int b) {
	 printf("%d\n",f(a, b));
}

int main(void) {
	printFunc(add, 1, 2);
	printFunc(sub, 1, 2);
	return 0;
}
\end{lstlisting}

\end{frame}

\begin{frame}{Mapping}
	Pointers to functions are often used for mapping. If you want to iterate through a list and call a function for every list item, you could do this with a new loop each time, but you also could write a function that takes a list and a function.\\
	\ \\
	If only we had an example to try this out...\\
	Oh wait... we do: The Dungeon.\\\ \\
	\begin{itemize}
		\item Write a mapping function that takes a list of monsters and a function.
		\item Edit the print\_monster\_list function to a print\_entity function.
		\item Now you can print the monsters list by calling the mapping function passing the list and the print\_entity function.
	\end{itemize}
\end{frame}

% nothing to do from here on
\end{document}
