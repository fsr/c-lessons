\input{slides_template}	% nothing to do here
\input{c_advanced_info} % TODO modify this if you have not already done so

% meta-information
\newcommand{\topic}{
	Bitoperations, storage classes, macros and function pointers
}

% nothing to do here
\title{\topic}
\supertitle{\course}
\date{}

% the actual document
\begin{document}

\maketitle

\begin{frame}{Contents}
	\tableofcontents
\end{frame}

\section{Bitoperations}
\subsection{}

\begin{frame}{About 0 and 1}
	As you know, your computer stores data in 0s and 1s.\\
	With C beeing a very low level language, we can manipulate data on bit layer.\\\ \\
	
	C offers the following bitoperations:\\\ \\
	
	\begin{tabular}{|l|c|l|}
																						  	  \hline
		\textbf{Operation} 	& \textbf{Syntax} 	& \textbf{Example} 							\\\hline
		Or 					& $|$ 				& $4\ |\ 1 = 0100\ |\ 0001 = 0101 = 5$ 		\\\hline
		And 				& $\&$ 				& $6\ \&\ 2 = 0110\ \&\ 0010 = 0010 = 2$ 	\\\hline
		Shift-Left 			& $<<$ 				& $5 << 2 = 00101 << 2 = 10100 = 20$ 		\\\hline
		Shift-Right 		& $>>$ 				& $10 >> 2 = 1010 >> 2 = 0010 = 2$ 			\\\hline
	\end{tabular}
\end{frame}

\section{Storage classes}
\subsection{}

\begin{frame}{ToDo}

\end{frame}

\section{Macros}
\subsection{}

\begin{frame}{ToDo}

\end{frame}

\section{Pointers to functions}
\subsection{}

\begin{frame}{Passing functions to functions}
	It is possible to pass a function as an argument of another function.\\
	Doing so, you have a function that can call different other functions.\\
	To be a parameter, there have to be information about the return type and the parameter list.\\
	\ \\
	Actually you do not pass a function but a pointer to that function.
\end{frame}

\begin{frame}[fragile]{Syntax}
	\begin{lstlisting}
#include <stdio.h>

int add(int a, int b) {
	return a + b;
}

int sub(int a, int b) {
	return a - b;
}

void printFunc(int (*f)(int, int), int a, int b) {
	 printf("%d\n",f(a, b));
}

int main(void) {
	printFunc(add, 1, 2);
	printFunc(sub, 1, 2);
	return 0;
}
\end{lstlisting}

\end{frame}

\begin{frame}{Mapping}
	Pointers to functions are often used for mapping. If you want to iterate through a list and call a function for every list item, you could do this with a new loop each time, but you also could write a function that takes a list and a function.\\
	\ \\
	If only we had an example to try this out...\\
	Oh wait... we do: The Dungeon.\\\ \\
	\begin{itemize}
		\item Write a mapping function that takes a list of monsters and a function.
		\item Edit the print\_monster\_list function to a print\_entity function.
		\item Now you can print the monsters list by calling the mapping function passing the list and the print\_entity function.
	\end{itemize}
\end{frame}

% nothing to do from here on
\end{document}
