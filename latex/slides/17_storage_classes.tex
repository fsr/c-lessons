\input{slides_template}	% nothing to do here
\input{c_advanced_info} % TODO modify this if you have not already done so

% meta-information
\newcommand{\topic}{
	Storage classes
}

% nothing to do here
\title{\topic}
\supertitle{\course}
\date{}

% the actual document
\begin{document}

\maketitle

\begin{frame}{Contents}
	\tableofcontents
\end{frame}
\section{Storage classes}
\subsection{}

\begin{frame}{Back to scopes}
	By introducing programs consisting of multiple files, we must differantiate between three cases to determine a variable's scope:
	\begin{enumerate}
		\item \textit{local}: only available in the block of its declaration
		\item \textit{global wrt. a file}: available to all functions in the file of its declaration
		\item \textit{global wrt. the program}: available to all functions in the program
	\end{enumerate}\bigskip
	While local variables get destroyed each time the program leaves their scope, all other variables are alive during the whole program execution.\\ \bigskip
	To modify this behaviour, C allows variable declarations with one of the following keywords: \textit{extern}, \textit{static}, \textit{register} and \textit{auto}.
\end{frame}

\begin{frame}[fragile]{\textit{extern}}
	In order to access a variable that has been defined outside the current file, you have to tell the compiler to look for it in another file.\\
	This is done by prefixing the variable declaration with the keyword \textit{extern}.\\
	\bigskip
	File io.c:
	\begin{lstlisting}
char line_buffer[512];
/* Some basic i/o functions */
\end{lstlisting}
	File parse.c:
	\begin{lstlisting}
extern char line_buffer[];
int parse_line(void) {
	/* Read from line_buffer in here */
}
\end{lstlisting}

\end{frame}

\begin{frame}{\textit{static}}
	\textit{Static} variables are placed on the data segment as if they were global.\\
	Their scope, however, depends on where they are declared:\\
	\bigskip
	\textit{Outside} any function
	\begin{itemize}
		\item They are global in the current file and cannot be accessed from anywhere else
		\item Thus you can have "private" variables and variables with the same names in different files 
	\end{itemize}
	\textit{Inside} a block
	\begin{itemize}
		\item They are local, but do not get destroyed when the block is left
		\item Thus you get functions with a "memory"
	\end{itemize}
\end{frame}

\begin{frame}[fragile]{\textit{register} and \textit{auto}}
	Variables that should be accessed fast can be put in \textit{register}s instead of the memory. However,
	\begin{itemize}
		\item They must not be larger than a word in your architecture
		\item You must not take addresses of such variables
		\item There is only a limited amount of registers, so the compiler may put the variables on the stack ignoring the \textit{register} keyword
	\end{itemize}
	\begin{lstlisting}
for (register i; i < 4096; ++i);	/* int is default */
\end{lstlisting}
	The compiler would do this during optimisation anyway.\\
	\bigskip
	All variables declared inside functions without any special keyword are \textit{auto}matically put on the stack $-$ no need to explicitely use \textit{auto}.
\end{frame}

\begin{frame}[fragile]{Storage classes of functions}
	By default, functions are global and available in the whole program.\\
	\bigskip
	You could declare them \textit{extern} in every file you want to call them, but it is better practice to include a header containing the function declaration.\\
	\bigskip
	Just like variables, \textit{static} functions have a "private" character and can be used to forbid calls from other files and avoid name clashes.
	\begin{lstlisting}
static void puts(char *str) {
	printf("\n%s\n", str);
}
\end{lstlisting}
	\bigskip
	It should be obvious that functions cannot be put into registers.	
\end{frame}

% nothing to do from here on
\end{document}
