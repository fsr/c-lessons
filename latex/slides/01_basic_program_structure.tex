\input{slides_template}	% nothing to do here
\usepackage[english]{babel}

\usepackage[utf8]{inputenc}

\newcommand{\course}{
	C introduction
}

\author{
	Richard Mörbitz,
	Manuel Thieme
}

\lstset{
	language = C,
	showspaces = false,
	showtabs = false,
	showstringspaces = false,
	tabsize = 4,
	escapechar = @
} % TODO modify this if you have not already done so

% meta-information
\usepackage{ulem}

\newcommand{\topic}{
    Basic program structure
}

\definecolor{orange}{RGB}{255,127,0}

% nothing to do here
\title{\topic}
\supertitle{\course}
\date{}

% the actual document
\begin{document}

\maketitle

\begin{frame}{Contents}
	\tableofcontents
\end{frame}

\section{Setup}
\subsection{}
\begin{frame}{OS's you may use}
	\begin{columns}[T]
		\column{.3\textwidth}
		Linux\\
		\uncover<2->{\textcolor{green}{recommended}}
		\column{.3\textwidth}
		Windows\\
		\uncover<3->{\textcolor{orange}{supported}}
		\column{.3\textwidth}
		Mac OS \textcolor<4->{red}{X}
	\end{columns}
\end{frame}
\begin{frame}[fragile]{Installing gcc on Linux}
	Ubuntu / Debian: 
	\begin{lstlisting}[numbers=none]
$ sudo apt-get install gcc
\end{lstlisting} \ \\ \ \\
	Arch:
	\begin{lstlisting}[numbers=none]
$ sudo pacman -S gcc
\end{lstlisting} \ \\ \ \\
	... and you're done ;-)
\end{frame}
\begin{frame}{cygwin}
	\begin{itemize}
		\item Download installer from \textit{https://cygwin.com/install.html}
		\item Run it
		\begin{itemize}
			\item "Install from Internet"
			\item Choose your installation path
			\item Choose path for installation files
			\item "Direct Connection"
			\item Choose a mirror
			\item Important software already is selected
			\item Optional: powerful editor "vim" in \textit{Editors}
			\item Watching loading bars...
			\item ???
			\item Profit!
		\end{itemize}
		\item Use cygwin-console like a linux terminal
	\end{itemize}
\end{frame}
\section{Hello World!}
\subsection{}
\begin{frame}[fragile]{The first program}
	\begin{itemize}
		\item Create a new file named \textbf{main.c}.
		\item Open it in your text editor of trust.
		\item Fill it as follows:
	\end{itemize}
	\begin{lstlisting}
#include <stdio.h>

int main(int argc, char* argv[]) {
	printf("Hello World!\n");
	/* Print "Hello World!" on the
	   command line */
	return 0;
}
\end{lstlisting}
\end{frame}
\begin{frame}[fragile]{From source to bits}
	\centering
	Source code\\\ \\
	$\Downarrow$\\\ \\
	\begin{lstlisting}[numbers=none]
$ gcc main.c
\end{lstlisting}
(Preprocessing, compiling, assembling, linking)
	\ \\\ \\
	$\Downarrow$\\\ \\
	Executable program (\textbf{a.out})\\\ \\
			\begin{lstlisting}[numbers=none]
$ ./a.out
\end{lstlisting}
\end{frame}
\section{Program structure}
\subsection{}
\begin{frame}[fragile]{A basic program}
	\begin{columns}[T]
		\column{.6\textwidth}
		\begin{lstlisting}
#include <stdio.h>

int main(int argc, char* argv[]) {

	printf("Hello World!\n");
	/* Print "Hello World!" on the
	   command line */

	return 0;
}
\end{lstlisting}
		\column{.4\textwidth}
		
		\ \\$\left. \begin{array}{c}\\\end{array}\right\rbrace $ Preprocessor statements
		\ \\\ \\$\left. \begin{array}{c}\\\\\\\\\\\\\end{array}\right\rbrace $ Main function
	\end{columns}
\end{frame}
\begin{frame}[fragile]{Preprocessor statements}	
	\begin{itemize}
		\item processed before compilation
		\item have their own language, start with a \textit{\#}
	\end{itemize}
	\begin{lstlisting}
#include <stdio.h>
\end{lstlisting}
	\begin{itemize}
		\item includes the \textbf{standard input/output library} (needed for printf, which is defined there)
		\item can also be used to define constants and much more, e.g.
	\end{itemize}
\begin{lstlisting}[numbers=none]
#define THE_ANSWER 42
\end{lstlisting}
\end{frame}
\begin{frame}[fragile]{The main function}
	\begin{itemize}
		\item basic function
		\item exists \textbf{exactly once} per program
		\item called on program start
	\end{itemize}
	\StartLineAt{3}
	\begin{lstlisting}
int main(int argc, char* argv[]) {
\end{lstlisting}
	\begin{itemize}
		\item as a function, \textit{main()} takes parameters
		\item get used to \textit{argc} and \textit{argv}, they will be explained later
		\item '$\lbrace$' marks the start of the main function scope
	\end{itemize}
\end{frame}
\begin{frame}[fragile]{The main function scope}
	\begin{itemize}
		\item contains all program statements
		\item they are processed from top to bottom
	\end{itemize}
	\StartLineAt{9}
	\begin{lstlisting}
	return 0;
}
\end{lstlisting}
	\begin{itemize}
		\item last statement, ends main function (and thus the whole program)
		\item \textit{0} tells the OS that everything went right
		\item '$\rbrace$' marks the end of the main function scope
	\end{itemize}
\end{frame}
\begin{frame}[fragile]{Statements}
	\begin{itemize}
		\item instructions for the computer
		\item end with a \textit{;} (semicolon)
	\end{itemize}
	\StartLineAt{5}
	\begin{lstlisting}
	printf("Hello World!\n");
\end{lstlisting} \ \\ \ \\
	\begin{itemize}
		\item there is the empty statement:
	\end{itemize}
	\begin{lstlisting}[numbers=none]
	;
\end{lstlisting}
	\begin{itemize}
		\item all statements are located in function blocks
	\end{itemize}
\end{frame}
\begin{frame}[fragile]{Comments}
	\begin{itemize}
		\item information for the programmer, cut out before compilation
	\end{itemize}
	Single line comments:
	\StartLineAt{6}
	\begin{lstlisting}
	// Print "Hello World!" on the command line
\end{lstlisting}
	Block comments (mutli-line):
	\StartLineAt{6}
	\begin{lstlisting}
	/* Print "Hello World!"
	   on the command line */
\end{lstlisting}
	Better use of block comments:
	\StartLineAt{6}
	\begin{lstlisting}
	/*
	 * Print "Hello World!"
	 * on the command line
	 */
\end{lstlisting}
\end{frame}
\section{Style}
\subsection{}
\begin{frame}[fragile]{A few words on style}
	\begin{itemize}
		\item {there can be multiple statements on one line}
		\item {intendation is not nessessary at all}
		\item<2-> BUT...
	\end{itemize}
	\ \\
	\begin{uncoverenv}<2->
	\begin{lstlisting}[numbers=none]
#include <stdio.h>
int
main	(int argc, char* argv[]){printf("Hello World!\n");
		// Prints
/*"Hello World!"			*/
		return 0;}
\end{lstlisting}
	\end{uncoverenv}
\end{frame}
\begin{frame}{Much more enjoyable}
	\begin{itemize}
		\item put each statement on a single line
		\item intend every statement in the main function by one tab / 4 \textit{spaces}
		\item use \textit{/* ... */} rather than \textit{// ...}
		\item write the main function arguments directly behind \textit{main}
		\item leave a \textit{space} between the closing ')' and the opening '\{'
	\end{itemize}
\end{frame}


% nothing to do from here on
\end{document}
