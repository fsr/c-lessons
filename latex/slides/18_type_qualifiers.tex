\input{slides_template}	% nothing to do here
\input{c_advanced_info} % TODO modify this if you have not already done so

% meta-information
\newcommand{\topic}{
	Type qualifiers
}

% nothing to do here
\title{\topic}
\supertitle{\course}
\date{}

% the actual document
\begin{document}

\maketitle

\begin{frame}{Contents}
	\tableofcontents
\end{frame}
\section{Type qualifiers}
\subsection{}

\begin{frame}[fragile]{Extended expressiveness}
To give more information about a variable to the compiler, you can\\
\textit{qualify} its type. There are three type qualifiers in C:\\
\begin{itemize}
	\item \textbf{const}, \textbf{volatile} and \textbf{restrict} (since C99)
\end{itemize}
\bigskip
The syntax is a bit complicated. Normally, a qualifier refers to the type to\\
its left, but the following is also valid (and more common!):
\begin{lstlisting}
const int a;				   /* equal to `int const a` */
\end{lstlisting}
Watch complex types:
\begin{lstlisting}
const int *foo;		/* mutable pointer, constant integer */
int const *foo;							/* same as above */
int * const foo;	/* constant pointer, mutable integer */
int const * const foo; 			  /* everything constant */
\end{lstlisting}
\end{frame}

\begin{frame}[fragile]{\textit{const}}
To prevent all write accesses to a variable, declare its type \textit{const}.\\
\begin{lstlisting}
void f(int *a);
...
const int i = 42;				  /* initialisation is ok */
i = 23; 	   /* error: assignment of read-only variable */
i++;			/* error: increment of read-only variable */
f(&i); /* warning: [...] discards 'const' qualifier [...] */
\end{lstlisting}
The warning above is useful pointer arguments with possible side effects.\\
\bigskip
The initialiser must be a constant expression, but:
\begin{lstlisting}
const int boardWidth; /* depends on runtime input */
...
*(int *) &boardWidth = read_input();
/* This hack is actually undefined behaviour */
\end{lstlisting}
\end{frame}

\begin{frame}[fragile]{\textit{volatile}}
If a variable's value may change between two accesses, we have to\\
prevent the compiler from optimising away subsequent reads or writes.\\
\bigskip
This is mainly used in low-level programming:
\begin{itemize}
	\item Hardware access (memory-mapped I/O)
	\item Threading (another thread modifies a value)
\end{itemize}
\bigskip
Example regarding optimisation:
\begin{columns}
\column{.5\textwidth}
	\begin{lstlisting}
#include <stdio.h>

int main(void) {
    int i = 42;
    printf("%d\n", i);
}
\end{lstlisting}
\column{.5\textwidth}
	\begin{lstlisting}
#include <stdio.h>

int main(void) {
    volatile int i = 42;
    printf("%d\n", i);
}
\end{lstlisting}
\end{columns}
\end{frame}

\begin{frame}[fragile]{\textit{volatile} ctd.}
After compilation with \textit{-O3}:
\begin{columns}
\column{.5\textwidth}
	\begin{lstlisting}[escapeinside={(*}{*)}]
main():
  sub    $0x8,%rsp
  mov    $0x2a,%esi
  mov    $0x400594,%edi
  xor    %eax,%eax
  callq  4003c0 <printf(*@*)plt>
  xor    %eax,%eax
  add    $0x8,%rsp
  retq   
  nopl   0x0(%rax)
\end{lstlisting}
\column{.5\textwidth}
	\begin{lstlisting}[escapeinside={(*}{*)}]
main():
  sub    $0x18,%rsp
  mov    $0x4005a4,%edi
  xor    %eax,%eax
  movl   $0x2a,0xc(%rsp) 
  mov    0xc(%rsp),%esi
  callq  4003c0 <printf(*@*)plt>
  xor    %eax,%eax
  add    $0x18,%rsp
  retq   
  nopw   %cs:0x0(%rax,%rax,1)
  nopl   (%rax)
\end{lstlisting}
\end{columns}
\bigskip
\pause
The compiler could not precompute \textit{i} and pass 42 to \textit{printf} directly.
\end{frame}

\begin{frame}[fragile]{\textit{restrict}}
If a pointer is \textit{restrict}, the compiler assumes that it is the only reference\\
to the targeted memory (no aliases). This enables further optimisation.\\
\bigskip
The most common appearance are function definitions:
\begin{lstlisting}[basicstyle=\scriptsize]
char *strcpy(char * restrict dest, const char * restrict src);
\end{lstlisting}
Note: pointers of different types are always assumed to be no aliases.\\
\bigskip
The compiler is unable to detect if \textit{restrict} pointers actually refer to different memory locations. This has to be ensured by the caller!\\
\bigskip
Non-pointer types cannot be qualified this way.
\end{frame}


% nothing to do from here on
\end{document}
