\input{slides_template}	% nothing to do here
\input{c_advanced_info} % TODO modify this if you have not already done so

% meta-information
\newcommand{\topic}{
	Parallel programming
}

% nothing to do here
\title{\topic}
\supertitle{\course}
\date{}

% the actual document
\begin{document}

\maketitle

\begin{frame}{Contents}
	\tableofcontents
\end{frame}

\section{Parallelism}
\subsection{}
\begin{frame}{Executing code in parallel}
    Each program has a process associated with it. At program start, this process has
    exactly one thread executing your \textit{main} function.\\
    \bigskip
    To achieve parallelism, you can
    \begin{itemize}
    		\item create a new process running the same code
    		\item call a function in a new thread
    \end{itemize}
    \bigskip
    In Unix systems, processes are created with the \textit{fork} system call.\\
    The new process will have its own memory to work with.\\
    For starting threads, libraries such as \textit{p[osix]threads} are used.\\
    All threads of a process share the same memory.
\end{frame}

\begin{frame}[fragile]{Use the fork}
    \begin{lstlisting}
#include <unistd.h>

int main(void) {
    pid_t pid = fork();

    if (pid @=@@=@ 0) {
        /* do stuff in child process */
    } else if (pid > 0) {
        /* do stuff in parent process */
    } else {
        /* fork failed */
        return 1;
    }
    return 0;
}
\end{lstlisting}
    Have a look at \textit{man 2 fork} for further information.
\end{frame}

\section{pthreads}
\subsection{}
\begin{frame}[fragile]{pthread\_create}
    To execute a function in a new thread, use:
    \begin{lstlisting}[numbers=none]
int pthread_create(pthread_t *thread,
                   const pthread_attr_t *attr,
                   void *(*start_routine) (void *),
                   void *arg);
\end{lstlisting}
    where
    \begin{itemize}
	    	\item *\textit{thread} is where the thread's id will be stored
	    	\item *\textit{attr} contains attributes for the thread (pass \textit{NULL} for default)
	    	\item \textit{start\_routine} is the function to execute. Both the single argument and the return value must be \textit{void *}.
	    	\item \textit{arg} is passed to the function to be used as an argument
    \end{itemize}
\end{frame}

\begin{frame}[fragile]{Hello threads}
    \begin{lstlisting}
#include <pthread.h>
#include <stdio.h>

void *hello_thread(void *tid) {
    printf("Hello, I am thread %d\n", *(int*) tid);
    pthread_exit(NULL);
}

int main(void) {
    pthread_t threads[5];
    for (int i=0; i < 5; ++i) {
       if (pthread_create(&threads[i], NULL,
                          hello_thread, (void *) i))
          return 1;
    }
    return 0;
}

\end{lstlisting}
\end{frame}
\begin{frame}[fragile]{How threads end:}
    \begin{itemize}
        \item \textit{start\_routine} returns
        \item \textit{pthread\_exit} is called
        \begin{lstlisting}[numbers=none]
void pthread_exit(void *retval);
\end{lstlisting}
        \item \textit{pthread\_cancel} is called from another thread
        \begin{lstlisting}[numbers=none]
int pthread_cancel(pthread_t thread);
\end{lstlisting}
        \item \textit{exit} is called from any thread (ending the process)
        \item \textit{main} finishes before \textit{start\_routine} (ending the process)
    \end{itemize}
\end{frame}
\begin{frame}[fragile]{Waiting for threads}
    To wait for a thread to finish, there is \textit{pthread\_join}
    \begin{lstlisting}[numbers=none]
int pthread_join(pthread_t thread, void **retval);
\end{lstlisting}
   
    \bigskip
    The thread passed to \textit{pthread\_join} must be joinable. For that we need the
    \textit{pthread\_attr\_t} structure. Here is how to initialise and destroy it:\\
        \begin{lstlisting}[numbers=none]
int pthread_attr_init(pthread_attr_t *attr);
int pthread_attr_destroy(pthread_attr_t *attr);
\end{lstlisting}
    \bigskip
    How to set/get the detachstate (pass \textit{PTHREAD\_CREATE\_JOINABLE}):
    \begin{lstlisting}[numbers=none]
int pthread_attr_setdetachstate(pthread_attr_t *attr,
                                int detachstate);
int pthread_attr_getdetachstate(const pthread_attr_t *attr,
                                int *detachstate);
\end{lstlisting}
\end{frame}

\begin{frame}[fragile]{bigger, better, joinable}
    \begin{lstlisting}[basicstyle=\scriptsize]
int main(void) {
    pthread_t threads[5];
    pthread_attr_t attr;
    
    pthread_attr_init(&attr);
    pthread_attr_setdetachstate(&attr, PTHREAD_CREATE_JOINABLE);
    
    for (int i=0; i < 5; ++i) {
       if (pthread_create(&threads[i], &attr,
                          hello_thread, (void *) i))
          return 1;
    }
    pthread_attr_destroy(&attr);

    void *st;
    for (int i=0; i < 5; ++i) {
       if (pthread_join(thread[i], &st))
           return 1;
       printf("Thread %d finished with %d\n", i, *(int *) st);
    }
    
    return 0;
}
\end{lstlisting}
\end{frame}

\section{Synchronisation}
\subsection{}
\begin{frame}[fragile]{Mutexes}
    Threads can communicate with each other by manipulating global variables or the value behind the \textit{arg} pointer we pass to \textit{pthread\_create}.\\
    To avoid race conditions, the pthread library provides mutexes.
    \begin{lstlisting}[numbers=none]
int pthread_mutex_destroy(pthread_mutex_t *mutex);
int pthread_mutex_init(pthread_mutex_t *restrict mutex,
                 const pthread_mutexattr_t *restrict attr);
pthread_mutex_t mutex = PTHREAD_MUTEX_INITIALIZER;
\end{lstlisting}
    \bigskip
    A mutex is a datatype that can be locked before and unlocked after accessing a variable.
    \begin{lstlisting}[numbers=none]
int pthread_mutex_lock(pthread_mutex_t *mutex);
int pthread_mutex_trylock(pthread_mutex_t *mutex);
int pthread_mutex_unlock(pthread_mutex_t *mutex);
\end{lstlisting}
\end{frame}

\begin{frame}[fragile]{Lock me if you can}
    \begin{lstlisting}[basicstyle=\tiny]
struct stuff {
    unsigned a;
    unsigned b;
}

struct stuff global = {1, 2};
pthread_mutex_t mutex;

void *thread_stuff(void *tid);
unsigned do_stuff(unsigned u);

int main(void) {
    pthread_t threads[5];
    pthread_mutex_init(&mutex, NULL);
    for (int i = 0; i < 5; ++i) {
        if (pthread_create(&thread[i], NULL, threadstuff, NULL))
            return 1;
    }
    
    pthread_mutex_destroy(&mutex)
    return 0;
}

void *thread_stuff(void *tid) {
    pthread_mutex_lock(mutex);
    unsigned a = do_stuff(global.a);
    global.b = a;
    pthread_mutex_unlock(mutex);
    
    pthread_exit(NULL);
}
\end{lstlisting}
\end{frame}

\begin{frame}[fragile]{Deadlocks incoming}
     \begin{lstlisting}[basicstyle=\scriptsize]
void *thread_1(void *tid) {
    pthread_mutex_lock(mutex_1);
    pthread_mutex_lock(mutex_2);
    /* stuff */
    pthread_mutex_unlock(mutex_1);
    pthread_mutex_unlock(mutex_2);
    
    pthread_exit(NULL);
}

void *thread_2(void *tid) {
    pthread_mutex_lock(mutex_2);
    pthread_mutex_lock(mutex_1);
    /* stuff */
    pthread_mutex_unlock(mutex_2);
    pthread_mutex_unlock(mutex_1);
    
    pthread_exit(NULL);
}
	\end{lstlisting}
	How to detect deadlocks is a part of the \textit{OS and security} lecture.
\end{frame}

% nothing to do from here on
\end{document}
